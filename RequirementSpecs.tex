\documentclass[12pt,a4paper]{article}
\usepackage{amsfonts}
\usepackage{amssymb}
\usepackage{graphicx}
\usepackage{bookmark}
\usepackage{hyperref}

\setlength{\parindent}{0em}

\begin{document}

\begin{titlepage}
	\begin{center}
		\begin{figure}[t]
			\centering
			\includegraphics[width=350px]{UP_Logo.png}
		\end{figure}
		
		\textsc{\LARGE COS301 Group Task 1 \newline\newline Requirements Specification for the NavUP System}
		
		\textbf{\newline Team Purple} \\
		\begin{flushright} \large
			Stephan Jack Nell \emph{u15124861} \newline
			Juan Jaques du Preez \emph{15189016} \newline
		\end{flushright}
		%\end{minipage}
		
		\vfill		
	\end{center}
\end{titlepage}

\tableofcontents
\newpage

\section{Introduction}
This document describes the Software Requirements Specification for the COS 301 Software Engineering Class project for 2017. We have been tasked to plan and create a system called NavUP. This system will serve as a guidance and basic navigation system for the 30000 people who enter the University of Pretoria everyday.
	\subsection{Scope}
	\subsection{Definitions, Acronyms, and Abbreviations}
		\begin{tabular}{|p{4cm}|p{10cm}|}
			\hline
				\textbf{Term} & \textbf{Description}\\
			\hline
				User & Someone who interacts with the mobile application.\\
			\hline
				Administrator/Admin & System administrator who is given specific permission for managing and controlling the application\\
			\hline
				Web-portal & A web application which present special facilities for admin.\\
			\hline
				GPS & Global Positioning System\\
			\hline
				GPS-Navigator & Installed software on a mobile phone which could provide GPS
connection and data, show locations on map and find paths from current position to defined destination.\\
			\hline
				Application Store & An installed application on a mobile phone which helps users to find new compatible applications with the mobile phone platform and download them from Internet.\\
			\hline
				API & Application Programming Interface\\
			\hline
				UI & User Interface\\
			\hline
		\end{tabular}
	\subsection{References}
		\begin{itemize}
			\item D. Kung, Object-oriented software engineering, 1st ed. New York: McGraw-Hill, a buss unit of the McGraw-Hill Companies, 2014, pp. 80-98.
		\end{itemize}
	\subsection{Overview}
\section{Overall Description}
NavUP main function will be to provide navigation to the user on the Hatfield Campus of the University of Pretoria. The navigation application will be accompanied with functions like, optimised route calculations based on shortest route and least congested route. NavUP will also allow the user to share location to a friend and provided navigation to a friend location on Campus.\newline

NavUP provides other functions with the intent of informing the user of all that the Campus can offer, NavUP will provide functionality to the user which will allow the user to view detailed information about the location the user is currently at or any location the user searched for. Furthermore, the user will be able to save frequent or favourite location to easily navigate to frequented locations
NavUP also focuses on entertainment and social side of the University of Pretoria by allowing the user to discover or create events that may occur at the many locations on Hatfield Campus.\newline

Crowdsourcing will play a pivotal role in the NavUP application user will be able to review buildings, restaurants or other services on Hatfield campuses, furthermore, the user will be able to report any problem on the campus which in turn will improve navigation for other users. The continued usage of the application by the users will allow the system to create heat maps of busy location or slow moving locations thus allowing even further optimisation to the navigation experience.\newline

The user will also be rewarded for their contribution to the NavUP application and will allow the user to play treasure hunt games or travel games, further an achievement system will allow the user to show off or compare between friends their level of contribution to the application.\newline
	\subsection{Product Perspective}
		NavUp is a product that is the first of its kind no system has yet been developed like NavUP, the navigation system will be built from the ground up to adapt to the specify need of the student, staff and guest of the University of Pretoria. However, NavUP will be making use of GIS technologies and databases to map Hatfield Campus grounds, these technologies will also be used to map the interiors of the buildings on Campus.\newline
		\subsubsection{User Interfaces}
			NavUP will provide a simplistic easy to use interface, with additional accessibility features for those with disabilities. The register and login interfaces will be minimalistic, yet descriptive and will be easy to use. Once logged in the user will encounter map interface which will serve as the home interface for NavUP.\newline

The Home interface will provide a large search bar on the top of the screen to search for location on the Hatfield Campus this search bar will also provide voice search functionality to accommodate for disability needs. When the search bar is selected two separate, buttons will emerge one will allow the user to display where the user is now on the Hatfield Campus the other will provide the options to read the location information. The Map will form part of two separate interface.\newline

A separate heat map button will be placed in the lower right-hand corner of the screen. If heat map is active a legend will appear where the search bar was and the colour of the map will change depending on the map congestion. The colour displayed on the heat map will also take into consideration people who suffer from colour blindness.\newline

A separate button will be in the lower left-hand corner of the screen. This button will lead to a different interface which will provide options to separate functionality like Account, Announcements, Events and Games.\newline

The Account interface will consist of radio buttons and checkboxes which will provide the user with a list preferences. The Announcements interface will be in a blog-like form and will display NavUP latest announcements. Games interface will indicate the user achievements in an upper banner, the reset of the interface will provide options to different games.\newline

Notification will make use of the device operating system to provide push notification to the user. The interface will remain the same across all devices whilst adapting the resolution to different screen sizes.\newline
		\subsubsection{Hardware Interfaces}
		\subsubsection{Software Interfaces}
		\subsubsection{Communications Interfaces}
		\subsubsection{Memory}
		\subsubsection{Operations}
		\subsubsection{Site Adaptation Requirements}
	\subsection{Product Functions}
	\subsection{User Characteristics}
	\subsection{Constraints}
	\subsection{Assumptions and Dependencies}
	
\section{Specific Requirements}
	\subsection{External Interface Requirements}
	\subsection{Functional Requirements}
	\subsection{Performance Requirements}
	\subsection{Design Constraints}
	\subsection{Software System Attributes}
	\subsection{Other Requirements}

\end{document}